\section{Taint Analysis}

Perhaps the most popular method to track flow of information throughout the program via program analysis is \emph{taint analysis}. The idea, stated simply, is to assign a \emph{taint tag} to every value resulting from a source statement, where the tag is merely a bit indicating that the value is security relevant. The taint tags are then propagated along the program's control-flow paths.

A main reason for the wide adoption of taint analysis as a framework for program-analysis-based information-flow security is its simplicity and scalability. Practical security products are required to scale to industry-level programs, which may consist of miillions of lines of code, and so an analysis that tracks minimal information is favorable.

At the same time, the limited expressiveness of taint analysis brings about accuracy challenges. 