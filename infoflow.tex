\chapter{Information-flow Security}

As discussed earlier, in Chapter \ref{XXX}, two of the main threats addressed by application security analysis are \emph{integrity vulnerabilities}, whereby untrusted data is able to reach security-sensitive operations without sufficient validation or sanitization, and \emph{confidentiality vulnerabilities}, whereby sensitive data is released to untrusted observers without sufficient anonymization (or declassification). Pleasingly, while these threats are dual to each other, they both reduce to the same formal problem statement.

\section{Problem Statement}

Consider a representation of the program as a graph, where the nodes are control locations and the edges denote control transitions. For the code example in Figure \ref{Fi:chIFS:example}, for instance, we obtain the graph in Figure \ref{Fi:chIFS:exampleGraph}. For conciseness, we represent control locations in the graph as pairs $(x,y)$, where the first (second) component $x$ ($y$) is the line number of the incoming (outgoing) statement. {\tt EN} and {\tt EX} are privileged entry and exit locations, respectively. 

\begin{figure*}
	\begin{minipage}[b]{0.5\textwidth}
		\begin{lstlisting}[numbers=left,basicstyle=\sffamily\small,language=Java,stepnumber=1]
b = nondet();
if (b) {
  // integrity source
  String input = getParameter(); 
  // integrity sink
  renderToHTML(input);
} else {
  // confidentiality source 
  String pwd = readPassword();
  // confidentiality sink
  Log.info("Password: "+pwd);
}
		\end{lstlisting}	
		\caption{\label{Fi:chIFS:example}Program in Java with integrity and confidentiality vulnerabilities}
	\end{minipage}
	\hspace{10pt}
	\begin{minipage}[b]{0.5\textwidth}
\xymatrix@C=0pt@R=20pt{
	& ({\tt EN})\ar[d]^{\tt nondet} & \\
	& (1,2)\ar[dl]_{\tt true}\ar[dr]^{\tt false}  \\
	(2,4)\ar[d]_{\tt getParameter} & & (2,9)\ar[d]^{\tt readPassword} \\
	(4,6)\ar[dr]_{\tt renderToHTML} & & (9,11)\ar[dl]^{\tt info} & \\
	& ({\tt EX}) &
}
		\caption{\label{Fi:chIFS:exampleGraph}The control-flow graph corresponding to the code in Figure \ref{Fi:chIFS:example}}
	\end{minipage}
\end{figure*}

Intuitively, in both integrity and confidentiality the checked property is whether there is a path extending between a \emph{source} node and a \emph{sink} node, which does not pass through a \emph{downgrader} node. Two such paths in the example in Figure \ref{Fi:chIFS:example} are $(2,4) \longrightarrow (4,6)$ (integrity vulnerability) and  
$(2,9) \longrightarrow (9,11)$ (confidentiality vulnerability). 

\subsection{The Principle of Noninterference}



\subsection{The Problem: Enforcement of Noninterference}



\section{Language-based Information-flow Security}

Language-based mechanisms are used to address different security threats. A notable example is the Java runtime system, which features bytecode verification \cite{XXX}, sandboxing \cite{XXX} as well as stack inspection \cite{XXX}. While bytecode verification requires a limited level of intra-procedural static program analysis, sanboxing and stack inspection rely solely on the Java language.

Language-based information-flow security aims to achieve exactly this. The core idea is to explode, or refine, ordinary types, like the Java {\tt String} and {\tt Integer}, into two parts. The first is the original type. The second part is a label that statically enforces security restrictions on the value via type checking, which is referred to as the \emph{security type} of the expression.

A security type system, therefore, is a set of security rules, which specify the type assigned to a program or expression given the types assigned to subexpressions. We illustrate some of the rules in Figure \cite{SabelfeldJournalPaper}.

\noindent {\bf TODO:} Add discussion of rules...

A major advantage of the language-based approach lies in its ability to account for implicit flows. This is done by assigning a label also to the program counter. Returning to the example in Figure \ref{Fi:implicitFlow}, we assign a high label to the conditional statement, which tests a high value, and consequently also to all the program counters dominated by the condition, namely the assignment to {\tt l}. This makes the implicit flow explicit. 

