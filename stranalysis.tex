String analysis is a form of static analysis that focuses on string variables in specific. The goal of the analysis is to approximate the set of runtime values that such variables can potentially be assigned at different points along the program's execution. String analysis has several important applications. Examples include detection of functional defects (e.g. checking whether format constraints on string inputs are enforced properly), code comprehension (e.g. computing the set of URLs that the client side of a web application is communicating with) and optimization (e.g. to batch together SQL queries or remove redundant input checks).

Among the main applications of string analysis are security verification and runtime enforcement. This is in light of the observation that inputs from remote soruces --- e.g. HTTP parameters in the case of web applications and data arguments passed as part of inter-application communication in the Android platform --- are serialized and received as string objects. This distinguishes strings as the most important target for security analysis.

\subsection{Theoretical Challenge}

Analogously to other forms of semantic static analysis, string analysis is an undecidable problem. This is not only for 
the inherent challenges posed by the programming language (loops, branching, etc), but also because of the complexity of approximating string values. 
 follows from the fact that the family of context-free languages (CFLs) is not closed under intersection, and so modeling string operations atop a CFL approximation of string variables is decidable only for emptiness checking, but not for inclusion and equivalence checking.